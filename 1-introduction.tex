- what is file fragment classification
- govdocs dataset benefits

Since the Govdocs1 dataset \cite{garfinkel_bringing_2009} was released, it became the default dataset used in file fragment classification research.

- studies comparison issues
-- source code
-- differente file types

But each research on the field chooses a different set of file types to compose its dataset. 

- contribution of this paper
- reference to thesis

This paper describes some of the experiments carried out during the development of the master's thesis ``Neural networks applied to file fragment classification''\cite{romero_atila_leites_neural_2020}. 

On that work, an important finding was obtained for future file fragment classification research.

That work started with the goal of finding a better model to classify file fragments, but 

the accuracy values initially obtained were much lower than those reported by other works. To find out why, 
the influence of number of classes and entropy were explored. 

- paper structure


To understand the impact that the choice of file types may have on the accuracy of a file fragment classification model, three experiments where conducted. The first uses a small neural network to almost replicate the results that other studies achieved using different models, by following the file type composition that each one used. The second investigates how similar each file type is to the others, to identify those that are easier to classify. The third uses this information to create datasets with easy to classify file types or hard to classify file types, showing how this different compositions affects the model final accuracy.